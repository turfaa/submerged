\documentclass[a4paper,titlepage]{article}
\usepackage[top=1in, bottom=1.25in, left=1.25in, right=1.25in]{geometry}

\parskip=10pt

\usepackage[scaled]{beramono}
\usepackage[T1]{fontenc}

\usepackage{amsmath}

\usepackage{listings}
\lstset{
	basicstyle=\footnotesize\ttfamily,
	tabsize=4,
	breaklines,
	frame=single,
	moredelim=[is][\underbar]{__}{__}
}

\usepackage{graphicx}
\usepackage{tikz}
\usepackage{float}
\usepackage{booktabs}

\renewcommand{\figurename}{Gambar}
\renewcommand{\refname}{Referensi}

\begin{document}

	\title{Laporan Tugas Besar 1 \\ IF2121 Logika Informatika}
	\author{
			Jonathan Christopher \\
			(13515001, K-01)
		\and
			Jordhy Fernando \\
			(13515004, K-01)
		\and
			Jauhar Arifin \\
			(13515049, K-01)
		\and
			Turfa Auliarachman \\
			(13515133, K-01)
	}
	\maketitle

	\section{Deskripsi Program}

		Program yang kelompok kami buat untuk Tugas Besar I mata kuliah Logika Informatika ini bernama \textit{Submerged}. \textit{Submerged} adalah sebuah game petualangan berbasis teks yang diimplementasikan menggunakan bahasa pemrograman deklaratif Prolog. Dalam game ini, pemain harus mengeksplorasi sebuah kapal selam yang sedang tenggelam dan mencari jalan keluar ke permukaan. Pemain dapat melakukan perintah-perintah untuk berpindah tempat, mengambil dan menggunakan objek, serta berinteraksi dengan NPC.

	\section{Keunikan dan Kelebihan}

		Game \textit{Submerged} ini memiliki beberapa keunikan:

		\begin{itemize}
			\item Game dikompilasi dari program Prolog menjadi sebuah file \textit{executable} yang sudah memuat \textit{interpreter} Prolog, sehingga game dapat langsung dijalankan tanpa terlebih dahulu menjalankan \textit{interpreter} Prolog secara manual.
			\item
		\end{itemize}

	\section{Peta Permainan dan Daftar Objek}

		\noindent Terdapat objek-objek statis maupun interaktif yang terletak pada beberapa lokasi dalam game ini:

		kegunaan, letak


	\section{Penjelasan \textit{Command}}

		\noindent Terdapat beberapa \textit{command} yang dapat digunakan oleh pemain dalam game ini:

		kegunaan, skenario penggunaan

	\section{Hasil Eksekusi}

		screenshot


	\section{Pembagian Kerja}

		\noindent Berikut adalah tabel pembagian tugas dalam kelompok:

		\begin{table}[H]
			\centering
			\begin{tabular}{@{}lll@{}}
				\toprule
				Komponen & Anggota & Dikerjakan pada \\ \midrule
				E & expression 			& keseluruhan ekspresi aritmatika \\
				T & term 				& ekspresi angka, hasil perkalian atau hasil pembagian \\

			\end{tabular}
		\end{table}

\end{document}
